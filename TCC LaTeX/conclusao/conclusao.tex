\chapter*{Conclus�o}
\addcontentsline{toc}{chapter}{Conclus�o} 

\section*{Considera��es Finais}
\addcontentsline{toc}{section}{Considera��es Finais} 

O presente trabalho trouxe uma abordagem da Computa��o Qu�ntica em n�vel introdut�rio. Procurou-se abordar os principais pr�-requisitos para tornar o texto autocontido. Procurou-se formular um material multidisciplinar, com enfoque tanto nos aspectos te�ricos como nos desdobramentos atuais e nas expectativas de mercado. O objetivo do texto � trazer um material de f�cil leitura para o iniciante nesse assunto, vindo de diversas �reas de Ci�ncias Exatas. 

\section*{Perspectivas}
\addcontentsline{toc}{section}{Perspectivas} 

Pretende-se dar continuidade ao material, acrescentando-se novos t�picos, de forma a torn�-lo uma refer�ncia �til aos iniciantes em Computa��o Qu�ntica. Em particular, pretende-se acrescentar uma introdu��o � Transformada de Fourier Qu�ntica e o Algoritmo de Shor, que � usado em um procedimento para encontrar fatores primos de um n�mero inteiro com um desempenho superior aos algoritmos cl�ssicos conhecidos. Tem-se tamb�m como perspectiva buscar algoritmos de interesse em problemas reais de Engenharia e �reas aplicadas.

Espera-se que em um futuro pr�ximo, os estudantes de gradua��o da UFSC possam ter como op��o em sua grade curricular um curso introdut�rio de Computa��o Qu�ntica e que esse material possa contribuir para alcan�ar esse objetivo. 